\documentclass[12pt,a4paper]{article}

% Codificación y tipografía
\usepackage[T1]{fontenc}
\usepackage[utf8]{inputenc}
\usepackage{lmodern}
\usepackage{microtype}

% Paquetes matemáticos y teoremas
\usepackage{amsmath,amssymb,amsthm}
\newtheorem{theorem}{Teorema}

% Gráficos y TikZ
\usepackage{graphicx}
\usepackage{tikz}

% Hipervínculos
\usepackage[hidelinks]{hyperref}

\title{Documento de prueba para extensión TeX}
\author{Prueba}
\date{\today}

\begin{document}
\maketitle

\section{Introducción}
Texto con caracteres Unicode: ñ, á, é, ü. Esto sirve para comprobar codificación y render.

\section{Matemáticas}
Ejemplo inline: \(e^{i\pi} + 1 = 0\).

Ejemplo en display:
\[
  I = \int_{0}^{\infty} e^{-x^2}\,dx = \frac{\sqrt{\pi}}{2}.
\]

\begin{theorem}[Ejemplo]
Sea \(f\) continua en \([a,b]\). Entonces \(f\) alcanza un máximo en \([a,b]\).
\end{theorem}

\section{Gráficos}
Figura generada con TikZ (compilar con pdflatex/xelatex):

\begin{center}
\begin{tikzpicture}[scale=1]
  \draw[->] (-0.5,0) -- (3.5,0) node[right] {$x$};
  \draw[->] (0,-0.5) -- (0,3.5) node[above] {$y$};
  \draw[domain=0:3,smooth,variable=\x,blue] plot ({\x},{0.5*\x*\x});
  \node at (2.5,3.2) {$y = 0.5 x^2$};
\end{tikzpicture}
\end{center}

\section{Referencias internas}
Vea el Teorema \ref{sec:theoremref} % ejemplo de referencia (no existe etiqueta real)

\end{document}